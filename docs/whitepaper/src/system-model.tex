\subsection{Transaction Ledger}\label{subsec:transaction-ledger}.
We adopt the definition of transaction ledger from~\cite{cryptoeprint:2016/889}.
A protocol $\Pi$ implements a robust transaction ledger provided that $\Pi$ is divided into blocks that determine the order in which transactions are incorporated into the ledger.
Blocks are assigned to time slots.
It should also satisfy the following properties:
\begin{enumerate}
    \item \textbf{Persistence.} Once a node of the system proclaims a certain transaction tx as stable, the remaining
    nodes, if queried, will either report tx in the same position in the ledger or will not report as stable any transaction in conflict to tx.
    Here the notion of stability is a predicate that is parameterized by a security parameter $k$; specifcally, a transaction is declared stable if and only if it is in a block that is more than $k$ blocks deep in the ledger.
    \item \textbf{Liveness.} If all honest nodes in the system attempt to include a certain transaction then, after the passing of time corresponding to u slots (called the transaction confirmation time), all nodes, if queried and responding honestly, will report the transaction as stable.
\end{enumerate}

\subsection{Time and Slots.}\label{subsec:time-and-slots.}
We consider a setting where time is divided into discrete units called slots.
A ledger, described in more detail above, associates with each time slot (at most) one ledger block.
Participants are equipped with (roughly synchronized) clocks that indicate the current slot.
This will permit them to carry out a distributed protocol intending to collectively assign a block to this current
slot.
In general, each slot $sl_r$ is indexed by an integer $r \subseteq \{1, 2, ..\}$, and we assume that the real
time window that corresponds to each slot has the following two properties:
\begin{enumerate}
    \item The current slot is determined by a publicly-known and monotonically increasing function of current time.
    \item Each participant has access to the current time.
    Any discrepancies between parties' local time are insignificant in comparison with the length of time represented by a slot.
\end{enumerate}

\subsection{Synchrony.}\label{subsec:synchrony.}
We consider an untrustworthy network environment that allows for adversarially-controlled message delays and immediate adaptive corruption.
Namely, we allow the adversary $A$ to selectively delay any messages sent by honest parties for up to $\Delta \subseteq \natural$ slots and corrupt parties without delay.

\subsection{Security Model.}\label{subsec:security-model.}
The system is composed of a set of nodes $N$.
Each node $n \in N$ is able to generate key-pairs ${(PK_n, SK_n)}$ without trusted public key infrastructure and is able to sign messages ${sign: (SK_n, m) \rightarrow \sigma}$.
Each node $n \in N$ is also able to verify signatures ${verify: (\sigma, PK_n, m) \rightarrow 0 | 1}$.
Each node $n \in N$ is associated with a unique wallet holding a balance of tokens $B_n$.

At any time $t$ a susbset ${V_t \subseteq N}$ of nodes is controlled by an adversary and are considered faulty.
Byzantine nodes can divert from the protocol and collude to attack the system.
The remaining honest nodes follow the protocol.
We assume that the total balance of all faulty nodes is less than 1/3 of the total balance $B$ of all nodes.

\subsection{Random Oracle.}\label{subsec:random-oracle.}
We also assume the availability of a random oracle.
This is a function $H: \{0,1\}^* \rightarrow \{0,1\}^\omega$ available to all parties that answers every fresh query with an independent, uniformly random string from f0; 1gw, while any repeated queries are answered consistently.

\subsection{External Systems.}\label{subsec:external-systems.}
We also assume multiple independent distributed systems ${S_1, \dots, S_k}$ with underlying ledgers ${L_1, \dots, L_k}$ as defined in~\cite{cryptoeprint:2019/1128}.
For each ledger there is a process $P_k$ that can influence the state evolution of the underlying ledger $L_k$ by committing a transaction $TX_k$ into it.
We extend the model defined in~\cite{cryptoeprint:2019/1128} by assuming that all ledgers allow for execution of simple predicates upon validation of transactions:
${verify: C \rightarrow 0 | 1}$, where $C$ is a \enquote{context} that contains description of state the transaction interacts with.
There is also a function ${desc: TX_k \rightarrow DESC^{TX_k}}$ that maps transaction $TX_k$ to some \enquote{description}, e.g.\ specifying the transaction value, recipient address, etc.
For each  $S_k$ there is a corresponding functionality unit $F_{S_k}$ that allows any $n$ equipped with the unit to interact with $S_k$.
Each node $n \in N$ is equipped with at least one such functionality unit and at most $k$ functionality units.