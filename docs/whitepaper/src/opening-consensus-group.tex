As long as Spectrum is an open-membership protocol, PBFTNetwork's assumption on a closed consensus group is nod valid.
Sybil attacks cat break any protocol with security thresholds such as PBFT's assumption that at most $f$ out of ${3 f + 1}$ members are honest, thus, an appropriate dynamic selection of opening consensus group is crucial for network livness and safety.
Consensus group members selection should be performed in a random and trusted way to ensure that a sufficient fraction of the selected members are honest, procedure itself should be independent of any internal or external advisers.
Bitcoin~\cite{nakamoto2009bitcoin} and many of its successors are using proof-of-work (PoW) to achieve this goal.
The consensus group selection problem is relevant to an equally important issue of the leader election.
In essence, PoW in Bitcoin is a robust mechanism that facilitates randomized selection of a \enquote{leader} that is eligible to produce a new block.
A primary consideration regarding PoW is the amount of energy required for the systems that operate on it.
A natural alternative to PoW is a mechanism that relies on the notion of proof-of-stake (PoS).
Rather than investing computational resources in order to participate in the leader election process, participants of a PoS system instead run a process that randomly selects one of them proportionally to the stake each possesses according to the current state of blockchain.

\subsubsection{Verifiable Random Function}

Verifiable Random Function (VRF) is a reliable way to introduce randomness in the protocol.
By definition the function $\mathcal{F}$ can be attributed to the VRF family if the following methods are defined for the $\mathcal{F}$:
\begin{enumerate}
    \item \textbf{Gen}: ${Gen(1^k) \rightarrow (pk, sk)}$, where $pk$ is the public key and $sk$ is the secret key;
    \item \textbf{Prove}: ${Prove(x; sk) \rightarrow (\mathcal{F}(x; sk), \pi)}$, where $x$ is a random input, $\mathcal{F}(x; sk)$ is a random value, sampled from $\{0,1\}^{l_{VRF}}$ and $\pi \vcentcolon= \pi(x; sk)$ is the associated proof;
    \item \textbf{Verify}: ${Verify(x, y, \pi; pk) \rightarrow 0 | 1}$, where the output is $1$ if and only if ${y=\mathcal{F}(x; sk)}$.
\end{enumerate}

There are several ways to design a VRF appropriate for our purposes.
At the first stage of the Spectrum protocol development we'll use an implementation, inspired by~\cite{cryptoeprint:2017/573}, which is based on the 2-Hash-DH verifiable oblivious PRF~\cite{asiacrypt:2014/233}.

This implementation is based on the hash functions $H$ and $H'$  with ranges $\{0,1\}^{l_{VRF}}$ and ${\langle g \rangle}$ respectively, where ${\langle g \rangle = q}$.
Thereby, public key and the private key received from the \textbf{Gen} output is ${pk = g \cdot k}$ and ${sk = H'(x) \cdot k}$ respectively.
\textbf{Prove} method returns the VRF generated random value ${y = H(x, sk)}$ and the proof ${\pi = (sk, EQDL(k : \log_{H'(x)}(sk) = \log_{g}(pk); x, pk))}$.
The \textbf{Verify} of ${(x, y, \pi, pk)}$ at first parses $\pi$ as ${(pk, \pi')}$ where $\pi'$ is a proof of equality of discrete logarithms and then verifies ${y = H(x, sk)}$ with the proof $\pi'$.
Finally, it returns $1$ if and only if both tests pass.
The proof notation ${EQDL(k : \log_{H'(x)}(sk) = \log_{g}(pk); x, pk)}$ stands for the string ${(c, s)}$ where $c = H(x, pk, g \cdot r, H'(x) \cdot r)$, ${s = r + k \cdot c}$ mod $q$, while the verification of ${(c, s)}$ on context ${x, pk}$ is performed by checking the equality
${c = H(x, pk, g^s \cdot pk^{-c}, H'(x)^s \cdot sk^{-c})}$.
The exact VRF simulator properties and functionality description can be found in the original Ouroboros Praos paper.

\subsubsection{Lottery protocol}
We will consider lottery protocol integrated in the dynamic PoS protocol flow.
There are some pre-defined primitives, detailed description of which can be also found in the original Ouroboros Praos paper.
\begin{enumerate}
    \item \textbf{Ideal Resettable Leaky Beacon} ${\mathcal_{F}}_{RLB}$: is used to sample epoch randomness from the blockchain.
    \item \textbf{Ideal Verifiable Random Function} ${\mathcal_{F}}_{VRF}$: generates random numbers.
    \item \textbf{Ideal Forward Secure Signature} ${\mathcal_{F}}_{KES}$: is a key evolving signature function which is used to sign a blocks and updates participant's public verification key $v_i^{kes}$ after every signing action.
    \item \textbf{Ideal Signature Scheme} ${\mathcal_{F}}_{DSIG}$: is a digital signature functionality for signing a transactions.
\end{enumerate}
Protocol initially is running by manually selected opening consensus group $\{PK_i\}_{i=0}^M$ of the predefined size $M$.
Stakeholders interact with each other and with the ideal functionalities ${\mathcal_{F}}_{RLB}$, ${\mathcal_{F}}_{VRF}$, ${\mathcal_{F}}_{DSIG}$, ${\mathcal_{F}}_{KES}$ over a sequence of $L = E \cdot R$ slots  ${S=\{sl_1,...,sl_L\}}$ consisting of $E$ epochs with $R$ slots each.
\begin{enumerate}
    \item \textbf{Initialization}.
    \begin{enumerate}
        \item All consensus group members i.e. $\forall PK_i, i \in M$ should generate the tuple of verification keys ${(v_i^{vrf}, v_i^{kes}, v_i^{dsig})}$, using the ideal functionalities ${\mathcal_{F}}_{VRF}$, ${\mathcal_{F}}_{KES}$, ${\mathcal_{F}}_{DSIG}$ instances, running on their machines.
        \item Then, to claim an initial stakes $\{s_i\}_{i=0}^M$ every protocol participant sends a request ${(\textbf{ver\_keys}, sid, PK_i, v_i^{vrf}, v_i^{kes}, v_i^{dsig})}$ to ${\mathcal_{F}}_{RLB}$, which saves the key tuple ${(PK_i, v_i^{vrf}, v_i^{kes}, v_i^{dsig})}$.
        \item Full set of the verification keys tuples ${V_{init} = \{(PK_i, v_i^{vrf}, v_i^{kes}, v_i^{dsig})\}_{i=0}^M}$ should be stored in the blockchain and acknowledged by all members of the initial consensus group.
        \item ${\mathcal_{F}}_{RLB}$, parameterized with confirmed $V_{init}$ is evaluated independently by every participant to sample an initial random seed value $\eta \leftarrow \{0, 1\}^\lambda$.
              Finally, all approved stakeholders should agree on the genesis block ${B_0=\left(\{(PK_i, v_i^{vrf}, v_i^{kes}, v_i^{dsig}, s_i)\}_{i=0}^M, \eta\right)}$.
    \end{enumerate}
    \item \textbf{Chain Extension}.
    After initialization, for every slot $sl \in S$, every online consensus group member $PK_i$ performs the following steps:
    \begin{enumerate}
        \item If a new epoch ${e_j \geqslant 2}$ has started $PK_i$ sends ${(\textbf{epochrnd\_req}, sid, PK_i, e_j)}$ to ${\mathcal_{F}}_{RLB}$ and receives $({\textbf{epochrnd}, sid, \eta_j)}$.
        \item Every online consensus group member collects en existed chains and verifying that for every chain every block, produced up to $Z$ blocks before contains correct data about slot $sl'$ leader $PK'$.
        To verify a valid slot leader, response from the ${\mathcal_{F}}_{VRF}$ to query ${(\textbf{Verify}, sid, \eta' || sl' || \textbf{test}, y', \pi', v^{vrf'})}$ should be ${(\textbf{Verified}, sid, \eta' || sl' || \textbf{test}, y', \pi', 1)}$ and $y'<T_j'$ as well.
        String \textbf{test} is an arbitrary and value $T_j'$ is a threshold for the stakeholder $PK'$ for the slot $sl'$.

        \textbf{Note:} at every slot $PK_i$ can be chosen as the slot leader with the probability ${p_i = \phi(\alpha_i, f) = 1-(1-f)^{\alpha_i}}$, where ${\alpha_i=s_i/\\\sum_{l=0}^{l=M} s_l}$ is a relative stake of the participant, fixed at the moment of the epoch $e_j$ start.
        Parameter $f$ is an active slots coefficient, responsible for percentage of slots in the epoch which should have at least one leader.
        In other words, it determines how many slots will pass before a new block is produced.
        Before the start of the epoch $e_j$ all consensus group participants updates their threshold values according to the blockchain snapshot ${T_i^j = 2^{l_{VRF}}\cdot \phi(\alpha_i^j, f)}$
        \item Then every lottery participant separately evaluates ${\mathcal_{F}}_{VRF}$ with his own inputs ${(\textbf{EvalProve}, sid, \eta_j || sl || \textbf{nonce})}$ and ${(\textbf{EvalProve}, sid, \eta_j || sl || \textbf{test})}$, where \textbf{nonce} is an arbitrary string.
        Received outputs ${(\textbf{Evaluated}, sid, y, \pi)}$ and ${(\textbf{Evaluated}, sid, \rho_y, \rho_\pi)}$ respectively includes generated random numbers ${y, \rho_y}$ and the associated proofs ${\pi, \rho_\pi}$.
        If ${y < T_i^j}$ then $PK_i$ is a slot leader. Random number $\rho_y$ will be used to sample a random seed for the next epoch.


        \item The ${\mathcal_{F}}_{VRF}$ is designed in such a way that not every slot has a leader, moreover, most of the slots remain empty to serve protocol synchronization.
        If there are $P$ several elected leaders for this slot, they all propose a new blocks $\{B_p\}_{p=0}^P$ with included proofs of the leadership ${(v_i^{vrf}, y, \pi)}$ and
        ${(\pi, \rho_\pi)}$. Block is signed with ${\mathcal_{F}}_{KES}$ by sending the request ${(\textbf{sign\_req}, sid, PK_i, B_p, sl)}$.
        Received from ${\mathcal_{F}}_{KES}$ response signature $\sigma_p$ is included in the proposed block.
        All forks will be further resolved according to the longest chain rule.
    \end{enumerate}

    \item \textbf{Consensus group reinitialization}.
    At the end of each epoch $e_j$, with $j \geqslant 1$ all consensus group lottery participants should commit their generated verification keys tuples as in the step (a) into the blockchain.
    If $PK_i$ keys are already stored in the blockchain, when he evaluates ${\mathcal_{F}}_{VRF}$ with a query ${(\textbf{EvalProve}, sid, \eta_j || e_j || \textbf{test*})}$ and compares the received random $y$ with a threshold ${T_i^j}^*$, which he sets according to $e_j$ snapshot with modified $\phi$ function (to select $M$ consensus group members out of $K$ total stakeholders).
    Afterwards he includes the associated proofs into the blockchain.
    If $PK_i$ is a new stakeholder, when he will be able to participate in the lottery only in the $j+1$ epoch.
    All previous participants, who has not provided new proofs of their consensus group membership are removed from the verified list.
\end{enumerate}


Using the scheme above we force opening consensus group members and slot leaders aren't publicly known ahead of time.
An attacker can't see who was a slot leader until after they have published a block, thus an attacker can't know
who specifically to attack in order to control a certain slot ahead of time.
Grinding attacks are also limited because an adversary can't arbitrarily control $\eta_j$ values.
All he can try to do is to make as many forks as possible to estimate the most advantageous, but according to the analysis this advantage doesn't change the security properties of the entire protocol.

Using VRF is also cheap and fast, only initialization requires communication between all participants to agree on a genesis block.
For large committee size this $O(n^2)$ communication complexity can be simplified to $O(n \cdot \log(n))$ with Merkle trees.
Single VRF evaluation is approximately 100 microseconds on x86-64 for a specific curves used in hash functions.
There is also great UC-extension for batch verification proposed by ~\cite{cryptoeprint:2022/1045} which make it even faster.
