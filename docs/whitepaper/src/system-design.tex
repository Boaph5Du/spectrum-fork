This section presents Spectrum protocol design starting from a naive approach based on PBFT and gradually addressing the challenges.

\subsection{Strawman Design: PBFTNetwork}\label{subsec:strawman-design}

For simplicity we begin with a notarization protocol based on PBFT, then iteratively refine it into Spectrum.

PBFTNetwork assumes that a group of ${n = 3f + 1}$ trusted nodes has been pre-selected upfront and fixed and at most $f$ of these nodes are byzantine.
At any given time one of these nodes is the \emph{leader}, who observes events on connected blockchains,
batch them and initiate round of notarization within the consensus group.
Remaining members of the consensus group verify the proposed batches by checking the presence of updates on corresponding blockchains.
Upon successful verification each node signs the batch with its secret key and sends the signature to the leader.

Under simplifying assumptions that at most $f$ nodes are byzantine the PBFTNetwork guarantees livness and safety.
However, the assumption of a fixed trusted committee is not realistic for open decentralized systems.
Moreover, as PBFT consensus members authenticate each other via non-transferable symmetric-key MACs, each consensus
member has to communicate with others directly, what results in $O(n^2)$ communication complexity.
Quadratic communication complexity imposes a hard limit on scalability of the system.
Such a design also scales poorly in terms of adding support for more chains.
The workload of each validator grows linearly with each added chain.

In the subsequent sections we address these limitations in four steps:
\begin{enumerate}
    \item \textbf{Opening consensus group and leaders.} We introduce a lottery-based mechanism for selecting consensus group and leaders dynamically.
    \item \textbf{Replacing MACs by Digital Signatures.} We replace MACs by digital signatures to make authentication transferable
    and thus opening the door for sparser communication patterns that can help to reduce the communication complexity.
    \item \textbf{Scalable Collective Signature Aggregation.} We utilize Byzantine-tolerant aggregation protocol that allows for
    quick aggregation of cryptographic signatures to reduce communication complexity to $O(\log n)$.
    \item \textbf{Eliminating Validator Bottleneck.} We shard consensus groups into units by the type of chain each node is able to handle.
\end{enumerate}

\subsection{Opening Consensus Group}\label{subsec:opening-consensus-group-and-leaders}
As long as Spectrum is an open-membership protocol, PBFTNetwork's assumption on a closed consensus group is nod valid.
Sybil attacks cat break any protocol with security thresholds such as PBFT's assumption that at most $f$ out of ${3 f + 1}$ members are honest, thus, an appropriate dynamic selection of opening consensus group is crucial for network livness and safety.
Consensus group members selection should be performed in a random and trusted way to ensure that a sufficient fraction of the selected members are honest, procedure itself should be independent of any internal or external advisers.
Bitcoin~\cite{nakamoto2009bitcoin} and many of its successors are using proof-of-work (PoW) to achieve this goal.
The consensus group selection problem is relevant to an equally important issue of the leader election.
In essence, PoW in Bitcoin is a robust mechanism that facilitates randomized selection of a \enquote{leader} that is eligible to produce a new block.
A primary consideration regarding PoW is the amount of energy required for the systems that operate on it.
A natural alternative to PoW is a mechanism that relies on the notion of proof-of-stake (PoS).
Rather than investing computational resources in order to participate in the leader election process, participants of a PoS system instead run a process that randomly selects one of them proportionally to the stake each possesses according to the current state of blockchain.

\subsubsection{Verifiable Random Function}

Verifiable Random Function (VRF) is a reliable way to introduce randomness in the protocol.
By definition the function $\mathcal{F}$ can be attributed to the VRF family if the following methods are defined for the $\mathcal{F}$:
\begin{enumerate}
    \item \textbf{Gen}: ${Gen(1^k) \rightarrow (pk, sk)}$, where $pk$ is the public key and $sk$ is the secret key;
    \item \textbf{Prove}: ${Prove(x; sk) \rightarrow (\mathcal{F}(x; sk), \pi)}$, where $x$ is a random input, $\mathcal{F}(x; sk)$ is a random value, sampled from $\{0,1\}^{l_{VRF}}$ and $\pi \vcentcolon= \pi(x; sk)$ is the associated proof;
    \item \textbf{Verify}: ${Verify(x, y, \pi; pk) \rightarrow 0 | 1}$, where the output is $1$ if and only if ${y=\mathcal{F}(x; sk)}$.
\end{enumerate}

There are several ways to design a VRF appropriate for our purposes.
At the first stage of the Spectrum protocol development we'll use an implementation, inspired by~\cite{cryptoeprint:2017/573}, which is based on the 2-Hash-DH verifiable oblivious PRF~\cite{asiacrypt:2014/233}.

This implementation is based on the hash functions $H$ and $H'$  with ranges $\{0,1\}^{l_{VRF}}$ and ${\langle g \rangle}$ respectively, where ${\langle g \rangle = q}$.
Thereby, public key and the private key received from the \textbf{Gen} output is ${pk = g \cdot k}$ and ${sk = H'(x) \cdot k}$ respectively.
\textbf{Prove} method returns the VRF generated random value ${y = H(x, sk)}$ and the proof ${\pi = (sk, EQDL(k : \log_{H'(x)}(sk) = \log_{g}(pk); x, pk))}$.
The \textbf{Verify} of ${(x, y, \pi, pk)}$ at first parses $\pi$ as ${(pk, \pi')}$ where $\pi'$ is a proof of equality of discrete logarithms and then verifies ${y = H(x, sk)}$ with the proof $\pi'$.
Finally, it returns $1$ if and only if both tests pass.
The proof notation ${EQDL(k : \log_{H'(x)}(sk) = \log_{g}(pk); x, pk)}$ stands for the string ${(c, s)}$ where $c = H(x, pk, g \cdot r, H'(x) \cdot r)$, ${s = r + k \cdot c}$ mod $q$, while the verification of ${(c, s)}$ on context ${x, pk}$ is performed by checking the equality
${c = H(x, pk, g^s \cdot pk^{-c}, H'(x)^s \cdot sk^{-c})}$.
The exact VRF simulator properties and functionality description can be found in the original Ouroboros Praos paper.

\subsubsection{Lottery protocol}
We will consider lottery protocol integrated in the dynamic PoS protocol flow.
There are some pre-defined primitives, detailed description of which can be also found in the original Ouroboros Praos paper.
\begin{enumerate}
    \item \textbf{Ideal Resettable Leaky Beacon} ${\mathcal_{F}}_{RLB}$: is used to sample epoch randomness from the blockchain.
    \item \textbf{Ideal Verifiable Random Function} ${\mathcal_{F}}_{VRF}$: generates random numbers.
    \item \textbf{Ideal Forward Secure Signature} ${\mathcal_{F}}_{KES}$: is a key evolving signature function which is used to sign a blocks and updates participant's public verification key $v_i^{kes}$ after every signing action.
    \item \textbf{Ideal Signature Scheme} ${\mathcal_{F}}_{DSIG}$: is a digital signature functionality for signing a transactions.
\end{enumerate}
Protocol initially is running by manually selected opening consensus group $\{PK_i\}_{i=0}^M$ of the predefined size $M$.
Stakeholders interact with each other and with the ideal functionalities ${\mathcal_{F}}_{RLB}$, ${\mathcal_{F}}_{VRF}$, ${\mathcal_{F}}_{DSIG}$, ${\mathcal_{F}}_{KES}$ over a sequence of $L = E \cdot R$ slots  ${S=\{sl_1,...,sl_L\}}$ consisting of $E$ epochs with $R$ slots each.
\begin{enumerate}
    \item \textbf{Initialization}.
    \begin{enumerate}
        \item All consensus group members i.e. $\forall PK_i, i \in M$ should generate the tuple of verification keys ${(v_i^{vrf}, v_i^{kes}, v_i^{dsig})}$, using the ideal functionalities ${\mathcal_{F}}_{VRF}$, ${\mathcal_{F}}_{KES}$, ${\mathcal_{F}}_{DSIG}$ instances, running on their machines.
        \item Then, to claim an initial stakes $\{s_i\}_{i=0}^M$ every protocol participant sends a request ${(\textbf{ver\_keys}, sid, PK_i, v_i^{vrf}, v_i^{kes}, v_i^{dsig})}$ to ${\mathcal_{F}}_{RLB}$, which saves the key tuple ${(PK_i, v_i^{vrf}, v_i^{kes}, v_i^{dsig})}$.
        \item Full set of the verification keys tuples ${V_{init} = \{(PK_i, v_i^{vrf}, v_i^{kes}, v_i^{dsig})\}_{i=0}^M}$ should be stored in the blockchain and acknowledged by all members of the initial consensus group.
        \item ${\mathcal_{F}}_{RLB}$, parameterized with confirmed $V_{init}$ is evaluated independently by every participant to sample an initial random seed value $\eta \leftarrow \{0, 1\}^\lambda$.
              Finally, all approved stakeholders should agree on the genesis block ${B_0=\left(\{(PK_i, v_i^{vrf}, v_i^{kes}, v_i^{dsig}, s_i)\}_{i=0}^M, \eta\right)}$.
    \end{enumerate}
    \item \textbf{Chain Extension}.
    After initialization, for every slot $sl \in S$, every online consensus group member $PK_i$ performs the following steps:
    \begin{enumerate}
        \item If a new epoch ${e_j \geqslant 2}$ has started $PK_i$ sends ${(\textbf{epochrnd\_req}, sid, PK_i, e_j)}$ to ${\mathcal_{F}}_{RLB}$ and receives $({\textbf{epochrnd}, sid, \eta_j)}$.
        \item Every online consensus group member collects en existed chains and verifying that for every chain every block, produced up to $Z$ blocks before contains correct data about slot $sl'$ leader $PK'$.
        To verify a valid slot leader, response from the ${\mathcal_{F}}_{VRF}$ to query ${(\textbf{Verify}, sid, \eta' || sl' || \textbf{test}, y', \pi', v^{vrf'})}$ should be ${(\textbf{Verified}, sid, \eta' || sl' || \textbf{test}, y', \pi', 1)}$ and $y'<T_j'$ as well.
        String \textbf{test} is an arbitrary and value $T_j'$ is a threshold for the stakeholder $PK'$ for the slot $sl'$.

        \textbf{Note:} at every slot $PK_i$ can be chosen as the slot leader with the probability ${p_i = \phi(\alpha_i, f) = 1-(1-f)^{\alpha_i}}$, where ${\alpha_i=s_i/\\\sum_{l=0}^{l=M} s_l}$ is a relative stake of the participant, fixed at the moment of the epoch $e_j$ start.
        Parameter $f$ is an active slots coefficient, responsible for percentage of slots in the epoch which should have at least one leader.
        In other words, it determines how many slots will pass before a new block is produced.
        Before the start of the epoch $e_j$ all consensus group participants updates their threshold values according to the blockchain snapshot ${T_i^j = 2^{l_{VRF}}\cdot \phi(\alpha_i^j, f)}$
        \item Then every lottery participant separately evaluates ${\mathcal_{F}}_{VRF}$ with his own inputs ${(\textbf{EvalProve}, sid, \eta_j || sl || \textbf{nonce})}$ and ${(\textbf{EvalProve}, sid, \eta_j || sl || \textbf{test})}$, where \textbf{nonce} is an arbitrary string.
        Received outputs ${(\textbf{Evaluated}, sid, y, \pi)}$ and ${(\textbf{Evaluated}, sid, \rho_y, \rho_\pi)}$ respectively includes generated random numbers ${y, \rho_y}$ and the associated proofs ${\pi, \rho_\pi}$.
        If ${y < T_i^j}$ then $PK_i$ is a slot leader. Random number $\rho_y$ will be used to sample a random seed for the next epoch.


        \item The ${\mathcal_{F}}_{VRF}$ is designed in such a way that not every slot has a leader, moreover, most of the slots remain empty to serve protocol synchronization.
        If there are $P$ several elected leaders for this slot, they all propose a new blocks $\{B_p\}_{p=0}^P$ with included proofs of the leadership ${(v_i^{vrf}, y, \pi)}$ and
        ${(\pi, \rho_\pi)}$. Block is signed with ${\mathcal_{F}}_{KES}$ by sending the request ${(\textbf{sign\_req}, sid, PK_i, B_p, sl)}$.
        Received from ${\mathcal_{F}}_{KES}$ response signature $\sigma_p$ is included in the proposed block.
        All forks will be further resolved according to the longest chain rule.
    \end{enumerate}

    \item \textbf{Consensus group reinitialization}.
    Every new epoch has a new consensus group.
    Any protocol participant can try to become a member of the consensus group if he is verified.
    Participant is verified if his generated verification key tuple is stored in the blockchain for a reliable period of time.

    At the end of each epoch $e_j$ with $j \geqslant 1$ all verified protocol participants
    evaluate ${\mathcal_{F}}_{VRF}$ with a query ${(\textbf{EvalProve}, sid, \eta_j || e_j || \textbf{test*})}$ and compares the received random $y$ value with a threshold ${T_i^j}^*$.
    Threshould which is calculated with the $\phi$ function, parameterized with a total stake at the beginning of the current epoch.
    Argument ${f = M \/ K}$, where $M$ is a number of new consensus group members to select and $K$ is the total number of the verified stakeholders.
    Afterwards he includes the associated proofs into the blockchain.
\end{enumerate}


Using the scheme above we force opening consensus group members and slot leaders aren't publicly known ahead of time.
An attacker can't see who was a slot leader until after they have published a block, thus an attacker can't know
who specifically to attack in order to control a certain slot ahead of time.
Grinding attacks are also limited because an adversary can't arbitrarily control $\eta_j$ values.
All he can try to do is to make as many forks as possible to estimate the most advantageous, but according to the analysis this advantage doesn't change the security properties of the entire protocol.

Using VRF is also cheap and fast, only initialization requires communication between all participants to agree on a genesis block.
For large committee size this $O(n^2)$ communication complexity can be simplified to $O(n \cdot \log(n))$ with Merkle trees.
Single VRF evaluation is approximately 100 microseconds on x86-64 for a specific curves used in hash functions.
There is also great UC-extension for batch verification proposed by ~\cite{cryptoeprint:2022/1045} which make it even faster.


\subsection{Replacing MACs by Digital Signatures}\label{subsec:replacing-macs-by-digital-signatures}

todo

\subsection{Scalable Collective Signature Aggregation}\label{subsec:scalable-collective-signature-aggregation}

todo

\subsection{Eliminating Validator Bottleneck}\label{subsec:eliminating-validator-bottleneck}
So far each member of consensus group had to track changes on all connected chains in order to participate in consensus properly.

\textbf{Observation 1:} Events coming from independent systems $S_k$ are not serialized.
Thus, the process of events notarisation can be parallelized.

\textbf{Observation 2:} Outbound transactions on independent systems $S_k$ can be independently signed.

Utilizing those properties we now introduce committee sharding.
We modify protocol in a way such that at each epoch $e$ $M$ distinct committees consisting of nodes equipped with functionality unit $F_{S_k}$ relevant to a specific connected chain $S_k$ are selected in a way described in (5.2.2).
All primitives and source of randomness are equal to different committees, the only difference is in the $f$ parameter of $\phi(\alpha_i, f)$ function, which is unique for every connected blockchain in order to guaranty expected number of members in every committee.
We denote one such committee shard as $V^{e}_{S_k}$, which uniquely maps to $S_k$.
Then, complete mapping of committees to chains at epoch $e$ can be represented as a set of tuples committee-chain $\{(V^{e}_{S_k}, S_k)\}$.
Throughout epoch $e$ all events and on-chain transactions on $S_k$ are handled exclusively by $V^{e}_{S_k}$.

Nodes in $V^{e}_{S_k}$ maintain a robust local ledger $L^{local}_k$ of notarized batches of events observed in $S_k$.

\subsubsection{Syncing Shards}

Notarized batches of events from local ledgers $\{L_i\}_{i=1}^{i=N}$ then should be synced in a super ledger $L^+$ in order for the system to be able to compute cross-chain state transition.
To facilitate this process batches of notarized events are broadcast to other committees.
The main actors at this stage are:
\begin{itemize}
    \item \textbf{Local leaders}: committees leaders, holding local notarized batches.
    \item \textbf{Relayers}: any protocol participant, who broadcasts notarized batches from \emph{Local leaders} to other committees' members.
    Every \emph{Local leader} can be a \emph{Relayer} at the same time.
    \item \textbf{General leader}: one of the \emph{Local leaders} who added a block consisted of all collected notarized batches to the $L^+$.
\end{itemize}

Since any \emph{Local leader} is able to publish his block to $L^+$ he can choose from two main strategies:
\begin{itemize}
    \item \textbf{Wait}: malicious strategy where \emph{Local leader} waits for broadcasts from other committees members and don't broadcast his own batch to eliminate competitors for adding a block.
    \item \textbf{Broadcast and wait}: fair strategy where \emph{Local leader} immediately broadcasts his batch, waits for broadcasts from committees members and honestly competes for adding a block.
\end{itemize}
Thus, there should be a motivation for individual \emph{Local leader} to choose the fair strategy instead of keeping his batch for too long.
This is achieved through the design of the incentive system.

There are three types of incentive: ${\{R_b, R_d, R_m\}}$, where $R_b$ is a guaranteed reward for adding a notarized batch to the block, $R_d$ is given for a broadcasting batch to the general leader and $R_m$ is given personally to the \emph{General leader} who mined the block.
Delivery reward $R_d$ is given to the \emph{Relayer} if and only if a delivery was made within a predetermined period of time $\Delta t_d$.
From the game-theoretic analysis, the following relationships between rewards were derived: ${R_b = 2 \cdot R_d, R_m = 3 \cdot R_d}$.
Thus, if ${R_d=0}$ there is no prior strategy for the \emph{Local leaders}, they will or wait for other batches either broadcast their batches with equal probability.
In case when ${R_d>0}$ it is distributed between the \emph{Local leader} and the \emph{Relayer}, i.e. ${R^l_d = \xi \cdot R_d}$ and  ${R^r_d = (1 - \xi) \cdot R_d}$, where ${\xi \in (0, 1)}$.
While ${\xi \rightarrow 1}$ probability that all \emph{Local leaders} will choose the \emph{Broadcast and wait} strategy approaches $1$.

As a result, the syncing Shards flow looks as follows:
\begin{itemize}
    \item Every \emph{Local leader} broadcasts (himself or through an intermediary as a \emph{Relayer}) his batch $b_i$, which contains the local notarization time $t^N_i$ and waits for batches from other \emph{Local leaders}.
    \item When waiting time approaches $\Delta t_d$, \emph{Local leader} forms a block from all collected batches ${\{b_i^j\}_{j=1}^{j=K}, K \le N}$ and add it to $L^+$.
    Block contains the set of the notarization times $\{t^{N^j}_i\}_{j=1}^{j=K}$ and block creation time $t^B_s$.
    \item After block is settled, all associated actors receive their rewards according to their roles: \emph{General leader} receives $R_m$, \emph{Local leaders}, whose batches are in the block receives $R_b$.
    In addition, if ${t^B_s - t^N_i^* < \Delta t_d}$, where $t^N_i^*$ is $t^N_i$ time, normalized to $L^+$ time, $i$-th committee \emph{Local leader} receives $R_d$ reward shared with the \emph{Relayer}.
\end{itemize}

\subsubsection{Forks and integrity}\label{subsec:resolving-forks}

Protocol flow implies that any of the local leaders can append their blocks to $L^+$, which leads to forks.
This type of fork is a normal part of the protocol lifecycle, however, total possible number of the normal forks in our protocol is much larger than in other blockchains, since there can also be a several local leaders in every connected $L_i$ committee.
The chance of occurring a malicious forks produced by adversary is minimized by lottery design.
In addition, the task for an adversary becomes more difficult by virtue of the interaction between the protocol participants during the Syncing Shards process.

The main rules for resolving forks are simple and are performed by the members of all committees when validating a proposed blocks:
\begin{itemize}
    \item \textbf{Max valid}: choose the longest appropriate chain given a set of valid chains that are available in the network.
    The depth of any block broadcast by a protocol member during the protocol must exceed the depths of any honestly-generated blocks from slots at least $K$ in the past.
    \item \textbf{Max stake}: if the \emph{Max valid} rule doesn't resolve a slot battle, then the valid chain chooses according to the stake size of the battled leaders, the maximum stake is the winner.
\end{itemize}

A large number of the normal forks, however, still significantly affect properties, that maintain the integrity of the $L^+$:
\begin{itemize}
    \item \textbf{Latency}: the number of elapsed slots required for a transaction to appear in a block on the $L^+$.
    \item \textbf{Finality}: the number of elapsed slots required for a transaction to become settled and immutable.
\end{itemize}
The Latency of the protocol is good enough due to the short duration of the slots.
Finality is guaranteed after $K_F$ slots, where $K_F$ is a pre-defined protocol parameter.
As a result of the functional features of our protocol, $F_F$ depends on the connected $L_i$ integrity properties.

Most ledgers do not guarantee instant finality of transaction, that means that any (or all) transactions may not be applied to corresponding $L_i$ ledgers in the end.
Different blockchains however has different Finality parameters, and time of elapsing $K_F$ should be longer than all of them.
Thus, the $K_F$ should be set with a margin and therefore using the number of slots $\Delta Sl$ that have passed in the Spectrum network, developers should be able to receive information about the number of blocks that have passed in all connected $L_i$ blockchains during this period of time.
The duration of the block in each $L_i$ is different, but the average values are preserved for a certain period of time ${\Delta T >> d_s}$, where $d_s$ is the duration of Spectrum's slot.
Thus, after each $\Delta T$ time interval, Spectrum network will update the set of constants: ${(\{d_{i}\}_{i=1}^{M},\{c_{i}\}_{i=1}^{M})}$, where $d_i$ is a block duration in the $L_i$, $c_i$ is the default reliable number of confirmations in the $L_i$, $M$ is the total number of the connected $L_i$.

Using the data above, each Spectrum's $\Delta Sl$ can be associated with the delta of blocks that have passed in any connected blockchain: ${\{\lfloor \Delta Sl \cdot d_s \mathbin{/} d_i)\rfloor\}_{i=1}^{N}}$.
When forming transaction, developers can specify a reliability factor $C$.
This factor will be compared with the ratio of the number of blocks passed on the associated $L_i$ blockchain to the default reliable number of confirmations $c_i$ of this network:
\begin{equation}
    \theta(i-L_i^{id})\cdot \left\{\frac{1}{c_i} \cdot \left\lfloor \Delta Sl \cdot \frac{d_s}{d_i}\right\rfloor\right\}_{i=1}^{M} >= C,\label{eq:equation2}
\end{equation}
where $\theta(x)$ is an indicator function which is 1 at $x = 0$, otherwise 0.


\subsection{Protocol Flow}\label{subsec:protocol-flow}

\subsubsection{Bootstrapping}\label{subsubsec:bootstrapping}

The system is bootstrapped in a trusted way.
A manually picked set of validators $V_0$ is assigned to the first epoch $e_0$.
On-chain vaults are initialized with an aggregated public key $aPK_0$ of the initial committee.
All initial committee members generate verification tuples ${(v_i^{vrf}, v_i^{kes}, v_i^{dsig})}$
and agree on the genesis block.

\subsubsection{Normal Flow}\label{subsubsec:normal-flow}

\begin{enumerate}
    \item Registration.
    All Spectrum stakeholders can register for becoming a committee member.
    To get a chance of becoming a member of $V_n$ in the epoch $e_n$ they register in a lottery during the $e_{n-2}$
    epoch by publishing their verification tuples ${(v_i^{vrf}, v_i^{kes}, v_i^{dsig})}$.
    \item Lottery.
    Once registration is done and epoch $e_{n-1}$ comes to the end, all registered participants evaluates
    ${\mathcal_{F}}_{VRF}$ locally and compare the generated random $y$ with their corresponding consensus threshold
    ${T_i^j}^*$ for this epoch.
    If successful, then publish $y$ and the associated proofs to form an approved consensus members table.
    \item Committee key aggregation.
    Once new committee is selected, nodes in $V_n$ aggregate their individual public keys $\{PK_i\}$ into
    a joint one $aPK_n$.
    \item Committee transition.
    Nodes in $V_{n-1}$ publish cross-chain message ${M_n : (aPK_n, \sigma_{n-1})}$ , where $aPK_n$ is
    an aggregated public key of the new committee $V_n$ , $\sigma_{n-1}$ is an aggregated signature of
    $M_n$ such that ${Verify(\sigma_{n-1}, aPK_{n-1}, Mn) = 1}$.
    Vaults are updated such that ${Vault\{(E_{n-1}, aPK_{n-1})\} \coloneqq (e_n, aPK_n)}$.
    \item Decentralized Asset Management (Custodial).
    Nodes in $V_n$ observe events on supported L1 chains, agree on the set of updates
    and compute state outbound state transitions accordingly.
    \item Notarisation (Non-custodial).
    Nodes in $V_n$ observe events on supported L1 chains, batch updates, collectively sign them and
    publish on-chain.
\end{enumerate}