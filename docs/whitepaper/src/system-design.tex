This section presents Spectrum protocol design starting from a naive approach based on PBFT and gradually addressing the challanges.

\subsection{Strawman Design: PBFTNetwork}\label{subsec:strawman-design}

For simplicity we bigin with a notarization protocol based on PBFT, then iteratively refine it into Spectrum.

PBFTNetwork assumes that a group of $n = 3f + 1$ trusted nodes has been pre-selected upfront and fixed and at most $f$ of these nodes are byzantine.
At any given time one of these nodes is the \emph{leader}, who observes events on connected blockchains,
batch them and initiate round of notarization within the consensus group.
Remaining members of the consensus group verify the proposed batches by checking the presense of updates on corresponding blockchains.
Upon successful verification each node signs the batch with its secret key and sends the signature to the leader.

Under simplifying assumptions that at most $f$ nodes are byzantine the PBFTNetwork guarantees livness and safety.
However, the assumption of a fixed trusted committee is not realistic for open decentralized systems.
Moreover, as PBFT consensus members authenticate each other via non-transferable symmetric-key MACs, each consensus
member has to communicate with others directly, what results in $O(n^2)$ communication complexity.
Quadratic communication complexity imposes a hard limit on scalability of the system.

In the subsequent sections we address these limitations in three steps:
\begin{enumerate}
    \item \textbf{Opening consensus group.} We introduce a lottery-based mechanism for selecting consensus group dynamically.
    \item \textbf{Replacing MACs by Digital Signatures.} We replace MACs by digital signatures to make authentication transferable
    and thus opening the door for sparser communication patterns that can help to reduce the communication complexity.
    \item \textbf{Scalable Collective Signaure Aggregation.} We utilize Byzantine-tolerant aggregation protocol that allows for
    quick aggregation of cryptographic signatures to reduce communication complexity to $O(\log n)$.
\end{enumerate}

\subsection{Opening Consensus Group}\label{subsec:opening-consensus-group}

todo

\subsection{Replacing MACs by Digital Signatures}\label{subsec:replacing-macs-by-digital-signatures}

todo

\subsection{Scalable Collective Signature Aggregation}\label{subsec:scalable-collective-signature-aggregation}

todo

\subsection{Protocol Flow}\label{subsec:protocol-flow}

\subsubsection{Bootstrapping}\label{subsubsec:bootstrapping}

The system is bootstrapped in a trusted way.
A manually picked set of validators $V_0$ is assigned to the first epoch $E_0$.
On-chain vaults are initialized with an aggregated public key $aPK_0$ of the inititial committee.

\subsubsection{Normal Flow}\label{subsubsec:normal-flow}

\begin{enumerate}
    \item Registration.
    Before an epoch starts, all Spectrum stakeholders can register for becoming an committee member.
    To get a chance of becoming a member of $V_n$ in the next epoch $E_n$ they register in a lottery
    by publishing their public keys $PK_c$ and locking collateral.
    \item Lottery.
    Once registration is done, nodes in $V_{n-1}$ compute $selectComettee: (C_n, R_n) \rightarrow V_n$,
    where  $C_n$ is a candidates pool, $R_n$ is a public random number.
    \item Committee key aggregation.
    Once new committee is selected, nodes in $V_n$ aggregate their individual public keys $\{PK_i\}$ into
    a joint one $aPK_n$.
    \item Committee transition.
    Nodes in $V_{n-1}$ publish cross-chain message $M_n : (aPK_n, \sigma_{n-1})$ , where $aPK_n$ is
    an aggregated public key of the new comettee $V_n$ , $\sigma_{n-1}$ is an aggregated signature of
    $M_n$ such that $Verify(\sigma_{n-1}, aPK_{n-1}, Mn) = 1$.
    Vaults are updated such that $Vault\{(E_{n-1}, aPK_{n-1})\} \coloneqq (E_n, aPK_n)$.
    \item Decentaralized Asset Management (Custodial).
    Nodes in $V_n$ observe events on supported L1 chains, agree on the set of updates
    and compute state outbound state transitions accordingly.
    \item Notarisation (Non-custodial).
    Nodes in $V_n$ observe events on supported L1 chains, batch updates, collectively sign them and
    publish on-chain.
\end{enumerate}