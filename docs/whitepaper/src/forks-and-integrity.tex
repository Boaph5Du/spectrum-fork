Protocol flow implies that any of the local leaders can append their blocks to $L^+$, which leads to forks.

This type of fork is a normal part of the protocol lifecycle, however, total possible number of the normal forks in our protocol is much larger than in other blockchains, since there can also be a several local leaders in every connected $L_i$ committee.
The chance of occurring a malicious forks produced by adversary is minimized by lottery design.
In addition, the task for an adversary becomes more difficult by virtue of the interaction between the protocol participants during the Syncing Shards process.

The main rules for resolving forks are simple and are performed by the members of all committees when validating a proposed blocks:
\begin{itemize}
    \item \textbf{Max valid}: choose the longest appropriate chain given a set of valid chains that are available in the network.
    The depth of any block broadcast by a protocol member during the protocol must exceed the depths of any honestly-generated blocks from slots at least $K$ in the past.
    \item \textbf{Max stake}: if the \emph{Max valid} rule doesn't resolve a slot battle, then the valid chain chooses according to the stake size of the battled leaders, the maximum stake is the winner.
\end{itemize}

A large number of the normal forks, however, still significantly affect properties, that maintain the integrity of the $L^+$:
\begin{itemize}
    \item \textbf{Latency}: the number of elapsed slots required for a transaction to appear in a block on the $L^+$.
    \item \textbf{Finality}: the number of elapsed slots required for a transaction to become settled and immutable.
\end{itemize}
The Latency of the protocol is good enough due to the short duration of the slots.
Finality is guaranteed after $K_F$ slots, where $K_F$ is a pre-defined protocol parameter.
As a result of the functional features of our protocol, $F_F$ depends on the connected $L_i$ integrity properties.

Most ledgers do not guarantee instant finality of transaction, that means that any (or all) transactions may not be applied to corresponding $L_i$ ledgers in the end.
Different blockchains however has different Finality parameters, and time of elapsing $K_F$ should be longer than all of them.
Thus, the $K_F$ should be set with a margin and therefore using the number of slots $\Delta Sl$ that have passed in the Spectrum network, developers should be able to receive information about the number of blocks that have passed in all connected $L_i$ blockchains during this period of time.
The duration of the block in each $L_i$ is different, but the average values are preserved for a certain period of time ${\Delta T >> d_s}$, where $d_s$ is the duration of Spectrum's slot.
Thus, after each $\Delta T$ time interval, Spectrum network will update the set of constants: ${(\{d_{i}\}_{i=1}^{M},\{c_{i}\}_{i=1}^{M})}$, where $d_i$ is a block duration in the $L_i$, $c_i$ is the default reliable number of confirmations in the $L_i$, $M$ is the total number of the connected $L_i$.

Using the data above, each Spectrum's $\Delta Sl$ can be associated with the delta of blocks that have passed in any connected blockchain: ${\{\lfloor \Delta Sl \cdot d_s \mathbin{/} d_i)\rfloor\}_{i=1}^{N}}$.
When forming transaction, developers can specify a reliability factor $C$.
This factor will be compared with the ratio of the number of blocks passed on the associated $L_i$ blockchain to the default reliable number of confirmations $c_i$ of this network:
\begin{equation}
    \theta(i-L_i^{id})\cdot \left\{\frac{1}{c_i} \cdot \left\lfloor \Delta Sl \cdot \frac{d_s}{d_i}\right\rfloor\right\}_{i=1}^{M} >= C,\label{eq:equation2}
\end{equation}
where $\theta(x)$ is an indicator function which is 1 at $x = 0$, otherwise 0.

