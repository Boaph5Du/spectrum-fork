\documentclass{article}
\usepackage[utf8]{inputenc}
\usepackage{mathtools}

\title{Spectrum: A decentralized programmable cross-chain network}
\author{Spectrum Labs}
\date{January 2023}

\begin{document}

    \maketitle

    \section{Introduction}\label{sec:introduction}

    Following the success of Bitcoin, many blockchain-based cryptocurrencies have been developed and deployed.
    To meet different requirements in various scenarios, a great number of heterogeneous blockchains have emerged.
    However, most present blockchain platforms are isolated systems.
    Therefore, interoperability between blockchains become one of the key issues
    that prevent the blockchain technology from wide adoption.


    \section{Model}\label{sec:model}

    todo

    \section{System Overview}\label{sec:system-overview}

    Spectrum is a standalone protocol that allows a large set of nodes to agree on:
    \begin{enumerate}
        \item A set of events coming from supported L1 chains
        \item A set of actions with on-chain assets managed by the Network
    \end{enumerate}

    Assets managed by the Network are stored on-chain in \emph{vaults}.
    Each vault stores epoch number $n$, an aggregated public key $aPK_n$ of the current validator set $V_n$ and
    is guarded with a script capable of performing verification of
    an aggregated signature $verify: (\sigma_n, m_n, aPK_n) \rightarrow 0 | 1$.

    \subsection{Spectrum's State Transition Function}\label{subsec:spectrum's-state-transition-function}

    Usually State Transition Function (STF) of a ledger looks like $apply: (S, T) \rightarrow S'$,
    where $S$ - current state of the ledger, $T$ - a set of transactions, $S'$ - resulting state of the ledger.
    Spectrum's STF as long as it operates partially on top of other ledgers, can be viewed
    as $apply: (S, S_O, T_I) \rightarrow (S', T_O)$, where $S$ -- current Spectrum's state,
    $S_O$ -- observed outbound state of connected ledgers, $T_I$ -- a set of inbound transactions,
    $S'$ -- resulting state of spectum's ledger, $T_O$ -- resulting set of outbound transactions
    that must be settled on connected L1s.

    \subsection{Bootstrapping}\label{subsec:bootstrapping}

    The system is bootstrapped in a trusted way.
    A manually picked set of validators $V_0$ is assigned to the first epoch $E_0$.
    On-chain vaults are initialized with an aggregated public key $aPK_0$ of the inititial comettee.

    \subsection{Protocol Flow}\label{subsec:protocol-flow}

    \begin{enumerate}
        \item Registration.
        Before an epoch starts, all Spectrum stakeholders can register for becoming an committee member.
        To get a chance of becoming a member of $V_n$ in the next epoch $E_n$ they register in a lottery
        by publishing their public keys $PK_c$ and locking collateral.
        \item Lottery.
        Once registration is done, nodes in $V_{n-1}$ compute $selectComettee: (C_n, R_n) \rightarrow V_n$,
        where  $C_n$ is a candidates pool, $R_n$ is a public random number.
        \item Comettee key aggregation.
        Once new comettee is selected, nodes in $V_n$ aggregate their individual public keys $\{PK_i\}$ into
        a joint one $aPK_n$.
        \item Comettee transition.
        Nodes in $V_{n-1}$ publish cross-chain message $M_n : (aPK_n, \sigma_{n-1})$ , where $aPK_n$ is
        an aggregated public key of the new comettee $V_n$ , $\sigma_{n-1}$ is an aggregated signature of
        $M_n$ such that $Verify(\sigma_{n-1}, aPK_{n-1}, Mn) = 1$.
        Vaults are updated such that $Vault\{(E_{n-1}, aPK_{n-1})\} \coloneqq (E_n, aPK_n)$.
        \item Decentaralized Asset Management (Custodial).
        Nodes in $V_n$ observe events on supported L1 chains, agree on the set of updates
        and compute state outbound state transitions accordingly.
        \item Notarisation (Non-custodial).
        Nodes in $V_n$ observe events on supported L1 chains, batch updates, collectively sign them and
        publish on-chain.
    \end{enumerate}

\end{document}