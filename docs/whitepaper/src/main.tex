\documentclass{article}
\usepackage[utf8]{inputenc}
\usepackage{mathtools}
\usepackage[
    backend=biber,
    style=authoryear,
    sorting=nyt,
    natbib
]{biblatex}
\usepackage[dvipsnames]{xcolor}
\usepackage{color}
\usepackage{csquotes}

\definecolor{SpectrumLavander}{HTML}{dadeed}
\addbibresource{references.bib}

\title{Spectrum: Cross-chain interoperability at scale}
\author{Spectrum Labs}
\date{February 2023}

\begin{document}
    \pagecolor{SpectrumLavander}
    \begin{sloppypar}

        \maketitle


        \section{Introduction}\label{sec:introduction}

        Following the success of Bitcoin, many blockchain-based cryptocurrencies have been developed and deployed.
        To meet different requirements in various scenarios, a great number of heterogeneous blockchains have emerged.
        However, most present blockchain platforms are isolated systems.
        Therefore, interoperability between blockchains become one of the key issues
        that prevent the blockchain technology from wide adoption.

        \subsection{Existing Solutions}\label{subsec:existing-solutions}

        \subsubsection{Centralized Solutions}\label{subsubsec:centralized-solutions}

        A classical solution to cross-chain interoperability is a trusted oracle that registers some event on one blockchain and performs the required action on the other.
        Centralized oracles provide fast and cheap transactions but lack a key feature, decentralization.
        The liquidity of a protocol built on this approach is custodial, which is a centralized approach similar to CeFi when users deposit their funds to an exchange's wallet:
        \begin{enumerate}
            \item A system is not sustainable when it depends on a single party
            \item If the oracle goes down unfinished swaps can appear frozen halfway
            \item A centralized oracle can censor transactions
            \item A malicious oracle can perform a man-in-the-middle attack by sending inaccurate data.
        \end{enumerate}

        \subsubsection{Semi-Centralized Solutions}\label{subsubsec:semi-centralized-solutions}

        Another common approach involves intermediate network consisting of fixed number of hand-picked oracles facilitating the transfer of data among multiple blockchains.
        PoA or PoS consensus is used with high collateral and a small (less than a hundred) number of nodes.
        Often, all funds transferred between blockchains are stored on some kind of threshold wallets, which are generated by the participants of the intermediate network, thus they are controlled by a fixed group of oracle operators.
        This method is inherently similar to the first one, although it is slightly more decentralized.


        \section{Related Work}\label{sec:related-work}

        todo


        \section{System Model}\label{sec:system-model}

        Spectrum operates in untrustworthy network environment that can arbitrary delay, re-order, drop or duplicate messages.
        To avoid the FLP impossibility the network is assumed to have weak synchrony property.
        The system is composed of a set of nodes $N$.
        Each node $n \in N$ is able to generate key-pairs ${(PK_n, SK_n)}$ without trusted public key infrastructure and is able to sign messages ${sign: (SK_n, m) \rightarrow \sigma}$.
        Each node $n \in N$ is also able to verify signatures ${verify: (\sigma, PK_n, m) \rightarrow 0 | 1}$.
        Each node $n \in N$ is associated with a unique wallet holding a balance of tokens $B_n$.

        At any time $t$ a susbset ${V_t \subseteq N}$ of nodes is controlled by an adversary and are considered faulty.
        Byzantine nodes can divert from the protocol and collude to attack the system.
        The remaining honest nodes follow the protocol.
        We assume that the total balance of all faulty nodes is less than 1/3 of the total balance $B$ of all nodes.

        We also assume multiple independent distributed systems ${S_1, \dots, S_k}$ with underlying ledgers ${L_1, \dots, L_k}$ as defined in~\cite{cryptoeprint:2019/1128}.
        For each ledger there is a process $P_k$ that can influence the state evolution of the underlying ledger $L_k$ by committing a transaction $TX_k$ into it.
        We extend the model defined in~\cite{cryptoeprint:2019/1128} by assuming that all ledgers allow for execution of simple predicates upon validation of transactions:
        ${verify: C \rightarrow 0 | 1}$, where $C$ is a \enquote{context} that contains description of state the transaction interacts with.
        There is also a function ${desc: TX_k \rightarrow DESC^{TX_k}}$ that maps transaction $TX_k$ to some \enquote{description}, e.g.\ specifying the transaction value, recipient address, etc.


        \section{Goals}\label{sec:goals}

        The resulted system should have the following properties:
        \begin{enumerate}
            \item \textbf{Decentralization.} The system should be decentralized.
            \item \textbf{Interoperability.} The system should be able to support a large number of heterogeneous blockchains.
            \item \textbf{Openness.} The system should allow anyone to participate in consensus permissionlessly.
            \item \textbf{Scalability.} The system should be able to operate normally while maintaining sufficiently large consensus groups.
            \item \textbf{Security.} The system should be able to withstand Sybil attacks.
            \item \textbf{Sustainability.} The system should be able to tolerate faults of the connected blockchains.
        \end{enumerate}


        \section{System Design}\label{sec:system-design}
        This section presents Spectrum protocol design starting from a naive approach based on PBFT and gradually addressing the challenges.

\subsection{Strawman Design: PBFTNetwork}\label{subsec:strawman-design}

For simplicity we begin with a notarization protocol based on PBFT, then iteratively refine it into Spectrum.

PBFTNetwork assumes that a group of ${n = 3f + 1}$ trusted nodes has been pre-selected upfront and fixed and at most $f$ of these nodes are byzantine.
At any given time one of these nodes is the \emph{leader}, who observes events on connected blockchains,
batch them and initiate round of notarization within the consensus group.
Remaining members of the consensus group verify the proposed batches by checking the presence of updates on corresponding blockchains.
Upon successful verification each node signs the batch with its secret key and sends the signature to the leader.

Under simplifying assumptions that at most $f$ nodes are byzantine the PBFTNetwork guarantees livness and safety.
However, the assumption of a fixed trusted committee is not realistic for open decentralized systems.
Moreover, as PBFT consensus members authenticate each other via non-transferable symmetric-key MACs, each consensus
member has to communicate with others directly, what results in $O(n^2)$ communication complexity.
Quadratic communication complexity imposes a hard limit on scalability of the system.
Such a design also scales poorly in terms of adding support for more chains.
The workload of each validator grows lineary with each added chain.

In the subsequent sections we address these limitations in four steps:
\begin{enumerate}
    \item \textbf{Opening consensus group.} We introduce a lottery-based mechanism for selecting consensus group dynamically.
    \item \textbf{Replacing MACs by Digital Signatures.} We replace MACs by digital signatures to make authentication transferable
    and thus opening the door for sparser communication patterns that can help to reduce the communication complexity.
    \item \textbf{Scalable Collective Signature Aggregation.} We utilize Byzantine-tolerant aggregation protocol that allows for
    quick aggregation of cryptographic signatures to reduce communication complexity to $O(\log n)$.
    \item \textbf{Eliminating Validator Bottleneck.} We shard consensus groups into units by the type of chain each node is able to listen.
\end{enumerate}

\subsection{Opening Consensus Group}\label{subsec:opening-consensus-group}

todo

\subsection{Replacing MACs by Digital Signatures}\label{subsec:replacing-macs-by-digital-signatures}

todo

\subsection{Scalable Collective Signature Aggregation}\label{subsec:scalable-collective-signature-aggregation}

todo

\subsection{Eliminating Validator Bottleneck}\label{subsec:eliminating-validator-bottleneck}

So far each member of consensus group had to track changes on all connected chains in order to participate in consensus properly.

\textbf{Observation 1:} Events coming from independent systems $S_k$ are not serilized.
Thus, the process of events notarisation can be parallelized.

\textbf{Observation 2:} Outbound transactions on independent systems $S_k$ can be independently signed.

Utilizing those properties we now introduce commettee sharding.
We modify protocol in a way such that at each epoch $e$ $M$ disinct commettees consisting of nodes equipped with functionality unit $F_{S_k}$ relevant to a specific connected chain $S_k$ are selected.
We denote one such commettee shard as $V^{e}_{S_k}$, which uniqiely maps to $S_k$.
Then, complete mapping of commettees to chains at epoch $e$ can be represented as a set of tuples commettee-chain $\{(V^{e}_{S_k}, S_k)\}$.
Throughout epoch $e$ all events and on-chain trasactions on $S_k$ are handled exclusively by $V^{e}_{S_k}$.

\subsubsection{Syncing Shards}

Once events from independent chains are notarized by dedicated commettees they have to be synced in a global log in order for the system to be able to compute cross-chain state transition.
To facilitate this process batches of notarized events are broadcast to other commettees.
Any commettee member is able to add notarized batch of events to the global log.
At each slot randomly selected shard leader from one of commettees has a greater chance of doing this, then comes other shard leaders, then regular validators.

\subsection{Protocol Flow}\label{subsec:protocol-flow}

\subsubsection{Bootstrapping}\label{subsubsec:bootstrapping}

The system is bootstrapped in a trusted way.
A manually picked set of validators $V_0$ is assigned to the first epoch $E_0$.
On-chain vaults are initialized with an aggregated public key $aPK_0$ of the initial committee.

\subsubsection{Normal Flow}\label{subsubsec:normal-flow}

\begin{enumerate}
    \item Registration.
    Before an epoch starts, all Spectrum stakeholders can register for becoming a committee member.
    To get a chance of becoming a member of $V_n$ in the next epoch $E_n$ they register in a lottery
    by publishing their public keys $PK_c$ and locking collateral.
    \item Lottery.
    Once registration is done, nodes in $V_{n-1}$ compute ${selectComettee: (C_n, R_n) \rightarrow V_n}$,
    where  $C_n$ is a candidates pool, $R_n$ is a public random number.
    \item Committee key aggregation.
    Once new committee is selected, nodes in $V_n$ aggregate their individual public keys $\{PK_i\}$ into
    a joint one $aPK_n$.
    \item Committee transition.
    Nodes in $V_{n-1}$ publish cross-chain message ${M_n : (aPK_n, \sigma_{n-1})}$ , where $aPK_n$ is
    an aggregated public key of the new committee $V_n$ , $\sigma_{n-1}$ is an aggregated signature of
    $M_n$ such that ${Verify(\sigma_{n-1}, aPK_{n-1}, Mn) = 1}$.
    Vaults are updated such that ${Vault\{(E_{n-1}, aPK_{n-1})\} \coloneqq (E_n, aPK_n)}$.
    \item Decentralized Asset Management (Custodial).
    Nodes in $V_n$ observe events on supported L1 chains, agree on the set of updates
    and compute state outbound state transitions accordingly.
    \item Notarisation (Non-custodial).
    Nodes in $V_n$ observe events on supported L1 chains, batch updates, collectively sign them and
    publish on-chain.
\end{enumerate}


        \section{Applications}\label{sec:applications}

        \subsection{Decentralized Cross-Chain Oracle}\label{subsec:cross-chain-oracle}
        In oracle mode of operation the system is capable of providing a notarized set of events observed on supported blockchains.
        Cross-Chain Oracle is a simple yet solving the cross-chain interoperability solution.

        \subsection{Custodial Asset Management}\label{subsec:custodial-asset-management}
        In custodial mode of operation the system is capable of managing user assets which are stored on corresponding blockchains in \emph{vaults}.
        Each vault stores epoch number $n$, an aggregated public key $aPK_n$ of the current validator set $V_n$ and
        is guarded with a script (smart-contract) capable of performing verification of
        an aggregated signature ${verify: (\sigma_n, m_n, aPK_n) \rightarrow 0 | 1}$.

        \paragraph{Natively Cross-Chain Applications}
        Decentralized custodial management in conjunction with a computational layer can be highly beneficial for expanding the capabilities of the system.
        Moving beyond simple bridges to what we call \emph{Natively Cross-Chain Applications} (NCCAs).
        NCCAs are applications that are deployed in cross-chain network and are capable of interacting with other blockchains without the need of external oracles or bridges.

        \newpage

        \printbibliography
    \end{sloppypar}

\end{document}